%!TEX root = Main.tex

%%%%%%%%%%%%%%%%%%%%%%%%%%%%%%%%%%%%%%%%%%%%%%%%%%%%%%%%%%%%%%%%%%%%%%%%%%%%%%%%
%  __  __                          _
% |  \/  |_   _   _ __   __ _  ___| | ____ _  __ _  ___  ___   _
% | |\/| | | | | | '_ \ / _` |/ __| |/ / _` |/ _` |/ _ \/ __| (_)
% | |  | | |_| | | |_) | (_| | (__|   < (_| | (_| |  __/\__ \  _
% |_|  |_|\__, | | .__/ \__,_|\___|_|\_\__,_|\__, |\___||___/ (_)
%         |___/  |_|                         |___/

\usepackage[utf8]{inputenc}
\usepackage{graphicx}
\usepackage[utf8]{inputenc}
\usepackage{setspace}
\usepackage{enumitem}

\pdfminorversion=4

% - - - - - - - - - - - - - - - - - - - - - - - -

% Review packages
\usepackage{soul}
\DeclareRobustCommand{\hlcyan}[1]{{\sethlcolor{cyan}\hl{#1}}}

% \usepackage{showframe}
% \usepackage[status=draft,layout=margin]{fixme}

% - - - - - - - - - - - - - - - - - - - - - - - -

\usepackage{amsmath,amssymb,amsfonts}
\usepackage{mathtools}
% Dependencies: ?
% Macros - mathtools: dcases
% Further definitions based on this package:
%
% [1] The following makes possible to create verticales lines inside matrix environments using the optional parameter [cc|c]
\makeatletter
\renewcommand*\env@matrix[1][*\c@MaxMatrixCols c]{%
    \hskip -\arraycolsep
    \let\@ifnextchar\new@ifnextchar
    \array{#1}}
\makeatother

% - - - - - - - - - - - - - - - - - - - - - - - -

% \usepackage{kbordermatrix}% http://www.hss.caltech.edu/~kcb/TeX/kbordermatrix.sty
\usepackage{blkarray}
\usepackage{physics}
% Dependencies: xparse, amsmath
% Macros: ..qty, pdv, dv,
% Further definitions based on this package:
%
% [1] \pqty should be adopt to the new matrix rules: \pmqty[]
\DeclareDocumentCommand\pmqty{ l m }{\begin{pmatrix}#1 #2 \end{pmatrix}}
%
% [2] Operators
\DeclareDocumentCommand\He{ l m }{\mathrm{He}\qty#1{#2}}
\DeclareDocumentCommand\Ve{ l m }{\mathbf{Ve}\pqty#1{#2}}
\DeclareDocumentCommand\Co{ l m }{\mathbf{Co}\pqty#1{#2}}
\DeclareDocumentCommand\Gr{ l m }{\mathbf{Gr}\pqty#1{#2}}
\DeclareDocumentCommand\ls{ l m }{\mathrm{Ls}\qty#1{#2}}
\def\vr{\varrho}
%
% [3] Linear fraction transformation symbols
\DeclareDocumentCommand{\Fu}{ s l >{\SplitArgument{4}{,}} m }{\IfBooleanTF{#1}{\sFimpl#2{u}#3}{\Fimpl#2{u}#3}}
\DeclareDocumentCommand{\Fl}{ s l >{\SplitArgument{4}{,}} m }{\IfBooleanTF{#1}{\sFimpl#2{l}#3}{\Fimpl#2{l}#3}}
\DeclareDocumentCommand{\Fimpl}{lmmmmmm}{\mathcal F_{#2}\qty#1{ \pmqty[c|c]{ #3 & #4 \\\hline #5 & #6 }, #7}}
\DeclareDocumentCommand{\sFimpl}{lmmmmmm}{\mathcal F_{#2}\qty#1{ \spmqty{ #3 & #4 \\ #5 & #6 }, #7}}

% - - - - - - - - - - - - - - - - - - - - - - - -

% \let\proof\relax \let\endproof\relax
\usepackage{amsthm}
% Dependencies: ?
% Macros: ...
% Further definitions based on this package:
%
% [1] New environment definitions
\theoremstyle{plain}
\newtheorem{theorem}{Theorem}
\newtheorem{corollary}{Corollary}
\newtheorem{lemma}[theorem]{Lemma}
\newtheorem{Def}[theorem]{Definition}
\newtheorem{prop}[theorem]{Proposition}
%
\theoremstyle{definition}
\newtheorem{remark}{Remark}
\newtheorem{example}{Example}
\newtheorem{problem}{Problem}
\newtheorem{assumptions}{Assumptions}
\newtheorem{assumption}{Assumption}

% - - - - - - - - - - - - - - - - - - - - - - - -

\usepackage{url}
\usepackage[
    hyperfootnotes=false,
    colorlinks=true,
    linkcolor=blue,
    urlcolor=blue,
    citecolor=blue,
    anchorcolor=blue,
    pagebackref=false]{hyperref}

% - - - - - - - - - - - - - - - - - - - - - - - -

% \usepackage{listings}
\usepackage{verbatim}

% - - - - - - - - - - - - - - - - - - - - - - - -

\usepackage{xparse}
\usepackage[breakable,skins]{tcolorbox}
\newlength{\pczoversizelength}
\setlength{\pczoversizelength}{8pt}
\DeclareDocumentEnvironment{pcb}{ s O{} O{blue} d<> D(){5} D(){20} D(){50} }
{

    \begin{tcolorbox}[
    enhanced,
    breakable,
    colback=\IfNoValueTF{#4}{#3}{#4}!#5,
    colbacktitle=#3!#6,
    colframe=#3!#7,
    subtitle style={boxrule=0.4pt,colback=#3!20},
    boxrule=1pt, arc=4pt, left=0.5\pczoversizelength, right=0.5\pczoversizelength, top=4pt, bottom=4pt,
    boxsep=0pt,
    toptitle=4pt,
    bottomtitle=4pt,
    fonttitle=\bfseries,
    coltitle=black,
    oversize,
    % =\pczoversizelength,
    #2]
}
{
    \end{tcolorbox}
}

\let\oldintercal\intercal
\def\intercal{{\protect\scalebox{0.7}{\protect\scalebox{0.8}[1]{$\top$}}}}

%%%%%%%%%%%%%%%%%%%%%%%%%%%%%%%%%%%%%%%%%%%%%%%%%%%%%%%%%%%%%%%%%%%%%%%%%%%%%%%%


\newenvironment{nocomment}{}{}